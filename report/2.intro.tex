\setlength{\parindent}{31pt}
\onehalfspacing
\chapter{Introdução}

O objectivo deste trabalho é o de desenvolver uma linguagem, implementando o interpretador, compilador e sistema de tipos.
Para além da implementação da linguagem \emph{Blaise} foi também traçado como objectivo implementar a linguagem \emph{O-Blaise}, incluíndo compilador e sistema de tipos. Adicionalmente, os extras que também se queriam implementar eram o da optimização das instuções \emph{box} e \emph{unbox}, apontadores, e das declarações de tipos.

Dos extras pretendidos apenas os apontadores não foram implementados devido a limitações de tempo. Quanto ao extra da optimização do \emph{box} e \emph{unbox} teve de ser imposta a restrição de não poder haver comentários no código compilado para evitar que ficassem entre estas instruções não sendo optimizadas pelo método implementado. Em relação à declaração de tipos, a funcionalidade está completamente implementada permitindo declaração de tipos recursivos.

Relativamente às linguagens \emph{Blaise} e \emph{O-Blaise} foram implementados o interpretador, o compilador e o sistema de tipos para ambas as linguagens.

